\section{Framework de Diseño}

El framework de diseño utilizado se basa en un proceso iterativo de
planteamiento, prototipado, implementación, y evaluación. Primero,
identificamos las necesidades y funcionalidades necesarias mediante una lluvia
de ideas y análisis de mesas de DJ reales/virtuales. Luego, creamos un
prototipo digital 2D para visualizar la interfaz y las posibles interacciones.
Luego de tener una vista amplia de lo que se necesita implementar, el
desarrollo se realiza en Unity, utilizando el SDK de interacción VR de Meta.
Finalmente, una etapa de playtesting y crítica nos permite identificar áreas de
mejora, las cuales se revisan en la siguiente etapa de desarrollo.

Con el objetivo proveer un UX óptimo y tener una ruta de diseño clara,
aplicamos los principios de diseño que fueron discutidos en clases.

\subsection{Principios de diseño}
\begin{itemize}
    \item \textbf{Signifiers}: Todos los elementos con los cuales el usuario debería poder interactuar
          están claramente indicados por colores, íconos, o formas que se utilizarán frecuentemente en el lenguaje visual de la aplicación.
          Con esto, pretendemos que el usuario pueda identificar rápidamente las funcionalidades disponibles, y acelerar el tiempo de aprendizaje.
    \item \textbf{Feedback}: Cada acción que el usuario realiza tendrá feedback auditivo (opcionalmente, visual). Girar perillas producirá ``clics'', los sliders
          desplazados por el usuario emitirán sonidos con un tono proporcional a su posición, y los botones sonarán y brillarán al ser presionados.
    \item \textbf{Mapping}: Los controles físicos de la mesa de mixing estarán configurados para seguir una dirección natural. Por ejemplo, una perilla en sentido
          horario incrementará el valor que controla (y viceversa). De forma similar, un slider que se mueva hacia arriba incrementará su valor.
    \item \textbf{Constraints}: Los controles como sliders y perillas tendrán límites físicos que impidan el movimiento. Al alcanzarlos, feedback visual y auditivo
          comunicará al usuario esta limitación.
\end{itemize}

\subsection{Heurísticas de Usabilidad}
\begin{itemize}
    \item \textbf{Visibilidad del estado del sistema}: La aplicación siempre mostrará información sobre lo que está ocurriendo. Por ejemplo,
          al cambiar de canción, se reproducirá un sonido de ``scratch'' característico, y la imagen del vinilo cambiará. Una barra de progreso también
          mostrará la posición del playback de la canción.
    \item \textbf{Coherencia y estándares}: El lenguaje visual de la aplicación será consistente en todas los botones, perillas, y slider, y usarán
          íconos similares a los que se ven en mesas de DJ reales. Además, toda interfaz compartira el mismo diseño visual y paleta de colores.
    \item \textbf{Prevención de errores}: Los vinilos que caigan fuera de la mesa de DJ serán teletransportados automáticamente a su posición original,
          evitando que el usuario pierda los vinilos.
    \item \textbf{Recuperación de errores}: Esta heurística será utilizada de forma escasa, dada la naturaleza física de los controles. Ejemplos particulares
          incluyen el brillo en rojo del botón de reproducción si no hay un vinilo colocado en la mesa.
    \item \textbf{Ayuda y documentación}: La aplicación tendrá un tutorial que enseñará al usuario las funcionalidades básicas. Este tutorial será opcional,
          y se podrá acceder a él desde el menú principal.
\end{itemize}

\subsection{¿Qué pueden lograr los usuarios?}
Los usuarios podrán vivir la experiencia de ser un DJ en un entorno VR.\@Las
funcionalidades incluyen:

\begin{itemize}
    \item Reproducción de música en tiempo real.
    \item Interacción con controles como perillas y sliders.
    \item Aplicación de efectos y filtros a las canciones.
    \item Reproducción de samples y loops complementarios.
    \item Grabación y reproducción de mezclas.
\end{itemize}

\subsection{Límites técnicos}

\begin{itemize}
    \item El uso de hand tracking impide el uso del feedback háptico de los controles,
          incrementando exponencialmente la importancia del feedback visual y auditivo.
    \item La precisión de la detección de BPM depende de los algoritmos implementados, y
          puede impactar la calidad de las mezclas, loops, y el rendimiento de la
          aplicación. Este último aspecto es importante, ya que la fluidez de la
          experiencia es indispensable en una aplicación VR.
    \item La implementación de parlantes y distintas mesas de DJ requiere de una
          arquitectura de software robusta y altamente modular, lo cual incrementa los
          tiempos de desarrollo.
\end{itemize}

\subsection{Profundidad de presencia}

Uno de los objetivos de la aplicación es maximizar la sensación de estar en un
festival musical. Para lograrlo, se emplearán assets y materiales que creen un
entorno colorido y dinámico, el cual se adapta a la música que el usuario está
mezclando. El uso de efectos visuales y auditivos, como luces y el sonido del
público, buscan maximizar la presencia del usuario en el entorno.