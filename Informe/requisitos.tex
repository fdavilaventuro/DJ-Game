\section{Requisitos de diseño y persona}

\subsection{Requerimientos Funcionales}

Al funcionar como un simulador de DJ en realidad virtual, la aplicación debe
contar con funcionalidades que simulan un equipo verdadero. Para lograrlo,
proponemos los siguientes requerimientos funcionales.

\begin{table}[h!]
    \centering
    \begin{tabular}{|c|p{8cm}|c|}
        \hline
        \textbf{N°} & \textbf{Descripción}                                                                                          & \textbf{Tipo} \\
        \hline
        RF01        & El usuario puede interactuar físicamente con el equipo de DJ (mezcladora, platos, perillas, etc.)             & Must have     \\
        \hline
        RF02        & El usuario puede reproducir y pausar música con botones en la mesa de DJ                                      & Must have     \\
        \hline
        RF03        & El usuario puede cambiar las canciones de los tracks a elección (precargadas o importables)                   & Must have     \\
        \hline
        RF04        & El usuario puede incrementar y disminuir el volumen de las canciones                                          & Must have     \\
        \hline
        RF05        & El usuario puede cambiar y transicionar entre canciones usando un crossfader                                  & Must have     \\
        \hline
        RF06        & El usuario puede aplicar filtros, como low pass, high pass, reverb, EQ, etc                                   & Must have     \\
        \hline
        RF07        & El usuario puede mezclar múltiples canciones a la vez                                                         & Should have   \\
        \hline
        RF08        & El usuario puede ajustar el tono y la velocidad de las canciones                                              & Should have   \\
        \hline
        RF09        & La aplicación debe tener un tutorial que explique las funcionalidades básicas                                 & Must have     \\
        \hline
        RF10        & El usuario puede grabar sus mezclas                                                                           & Nice to have  \\
        \hline
        RF11        & El setup de DJ debe incluir distintos parlantes, los cuales pueden ser conectados a distintas mesas de mezcla & Nice to have  \\
        \hline
        RF12        & El usuario puede girar los discos para controlar las canciones                                                & Nice to have  \\
        \hline
        RF13        & La aplicación puede detectar el BPM de las canciones                                                          & Nice to have  \\
        \hline
        RF14        & El usuario puede utilizar samples y loops que se ajustan al BPM de las canciones                              & Nice to have  \\
        \hline
        RF15        & El usuario puede cambiar el BPM de las canciones en tiempo real                                               & Nice to have  \\
        \hline
        RF16        & El usuario puede elegir distintos ambientes para desempeñar su set de DJ                                      & Nice to have  \\
        \hline
        RF17        & El usuario puede afectar al público dependiendo de la calidad de su set (ej. animaciones, reacciones)         & Nice to have  \\
        \hline
    \end{tabular}
    \caption{Requerimientos funcionales}
\end{table}

\newpage{}
\subsection{Requerimientos no Funcionales}

De forma similar, planteamos requerimientos no funcionales, los cuales ayudarán
a mejorar la experiencia de usuario.
\begin{table}[h!]
    \centering
    \begin{tabular}{|c|p{8cm}|c|}
        \hline
        \textbf{N°} & \textbf{Descripción}                                                             & \textbf{Tipo} \\
        \hline
        RNF01       & La aplicación debe ejecutarse a un mínimo de 60 FPS                              & Must have     \\
        \hline
        RNF02       & La aplicación no debe tener retrasos perceptibles en el feedback de las acciones & Must have     \\
        \hline
        RNF03       & La aplicación debe ser utilizable sin experiencia previa como DJ                 & Should have   \\
        \hline
    \end{tabular}
    \caption{Requerimientos no funcionales}
\end{table}

\subsection{Audiencia}
La audiencia objetivo de la aplicación son adolescentes y adultos jóvenes de
entre 14 y 30 años con un gusto por la música y el mixing. Estas personas
suelen tener mayor afinidad por géneros musicales como, como EDM, house, etc.,
los cuales también son comúnmente utilizados en eventos sociales. Por último,
los usuarios en este rango de edad suelen tener experiencia previa con
tecnología y videojuegos, lo cual puede facilitar su adaptación a los controles
de una aplicación VR.\@

\subsection{Especificación de Usuario Final (User Person)}
\begin{itemize}
    \item \textbf{Nombre:} Alejandro Casquino
    \item \textbf{Edad:} 27 años
    \item \textbf{Ocupación:} Estudiante de arquitectura
    \item \textbf{Habilidad tecnológica:} Nivel intermedio/alto. Suele jugar videojuegos, con o sin experiencia previa en VR.
    \item \textbf{Intereses:} Fiestas, música electrónica, festivales, tecnología
    \item \textbf{Motivación:} Aprender a mezclar música y divertirse con una aplicación de realidad virtual.
    \item \textbf{Frustraciones:} Difícil acceso a equipos de DJ profesionales
    \item \textbf{Comportamiento:} Juega videojuegos regularmente, tanto móviles como en PC. Asiste a eventos musicales de forma regular, y tiene curiosidad por la tecnología VR.
\end{itemize}

\subsection{Objetivo de la Experiencia}
Al finalizar el desarrollo de la aplicación, nuestro objetivo es obtener un
simulador de DJ en realidad virtual que permita a los usuarios sin experiencia
previa en DJ aprender habilidades básica de mixing y divertirse con la música
en un entorno immersivo.

\subsection{Narrativa}
Al ponerse el Meta Quest 2, Alejandro se encuentra en el centro de una cabina
de DJ iluminada por luces de colores neón, al frente de un público virtual que
baila al ritmo de la música. Frente a él, la mesa brilla con botones, vinilos y
sliders. Aunque está solo en su habitación, la realidad virtual lo transporta a
un festival, donde puede sentir el bajo de los parlantes y las voces del
público.

La experiencia comienza cuando Alejandro elige su primera canción y la mezcla
con otra, aplicando filtros y efectos. Cada cambio que hace modifica la música
de forma inmediata. Gracias a los controles fáciles de usar, se puede explorar
nuevas técnicas de mixing, girar los discos y ajustar el volumen como si
estuviera en la cabina, sintiendo la emoción propia de un evento de verdad.

A medida que la sesión avanza, Alejandro prueba diferentes géneros musicales.
Además, un tutorial le ayuda a dominar nuevas funciones, permitiéndole aprender
nuevas técnicas. Luego de unas horas, Alejandro no sólo aprendió a mezclar
música, sino que ha vivido una experiencia inmersiva y divertida desde su
habitación.

\subsection{Arco de la Historia}

\begin{itemize}
    \item \textbf{Introducción:} El usuario abre la aplicación, y se encuentra en el backstage de un festival musical. Se escucha la música de fondo y la multitud afuera.
    \item \textbf{Inicio:} El usuario se dirige al escenario principal, donde se encuentra la cabina de DJ.\@ Al entrar, un breve tutorial le da la bienvenida y explica las funciones principales.
    \item \textbf{Adaptación:} El usuario selecciona su primera canción y experimenta moviendo los controles para mezclarla y aplicar efectos. La multidud virtual reacciona, animando al usuario a seguir.
    \item \textbf{Desarrollo:} El usuario comienza a experimentar con diferentes técnicas de mezcla, utilizando los efectos y controles disponibles en la cabina de DJ.\@
    \item \textbf{Clímax:} El usuario se siente completamente inmerso en la experiencia, como si estuviera en un verdadero festival de música. Las luces y el público brillan y saltan al ritmo de la música.
    \item \textbf{Fin:} El usuario termina su sesión de mezcla, luego de grabar su pista final. Finalmente, regresa al backstage, acompañado de música que disminuye gradualmente.
\end{itemize}

\newpage{}
\subsection{Métrica HEART}

\begin{table}[h!]
    \centering
    \begin{tabular}{|c|p{5cm}|p{5cm}|}
        \hline
        \textbf{Métrica} & \textbf{Descripción}                                                                                                  & \textbf{Indicador}                                               \\
        \hline
        Happiness        & Encuestas dentro de la aplicación, así como medición de movimiento de los controles y la cámara                       & Calificación de satisfacción y movimiento frecuente de controles \\
        \hline
        Engagement       & Cantidad de tiempo que los usuarios pasan en la aplicación, y la cantidad de canciones distintas utilizadas en el mix & Tiempo de uso medio y variedad de canciones                      \\
        \hline
        Adoption         & Número de personas que utilizan la aplicación por primera vez                                                         & Nuevos usuarios por mes                                          \\
        \hline
        Retention        & Número de usuarios que prueban la aplicación en un periodo de una semana luego del uso inicial                        & Usuarios recurrentes por semana                                  \\
        \hline
        Task Success     & El usuario debe poder aplicar al menos 3 efectos diferentes sin problemas durante una sesión                          & Porcentaje de éxito mayor al 90\%                                \\
        \hline
    \end{tabular}
    \caption{Métrica HEART}
\end{table}